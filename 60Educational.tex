%% !TEX root = CAREER II.tex

\section{Integrated Educational Activity}

\paragraph{Course Creation:}


\begin{itemize}[noitemsep,leftmargin=*]
\vspace{-0.2cm}
\item [C1] PI Mueen has developed the first data mining course (CS 521) at UNM. 
The course attracts a significant number of students from various departments. However, UNM does not have any course specializing in web search and mining. PI Mueen proposes to develop a course covering the core concepts including information retrieval and ranking, online advertisements, recommender systems and web crawling. The proposed research activities involve a significant exploration of social media mining that will be integrated into the assignments and projects of this course. IDEA scores will be used for evaluation.


\item[C2] PI Mueen proposes to create a sequence of video tutorials on data analytics for high school students and teachers. The tutorials will be distributed through the NSF funded New Mexico Computer Science for All initiative \cite{cs4all}. This tutorial will elucidate simple data analysis tasks such as charting, smoothing, filtering, making frequency distributions, etc. The NM-CSforAll is a very successful program to evangelize computer science to both students and teachers. The success of the tutorials will be assessed in the yearly summer camp of the NM-CSforAll (letter from Dr. Moses is attached).
\end{itemize}


\paragraph{Outreach:}  
%We will have three outreach activities directly related to the proposed research works.
\vspace{-0.1cm}
\begin{itemize}[noitemsep,leftmargin=*]
\vspace{-0.2cm}
\item [O1] UNM's University Communication and Marketing (UCAM) department and the Media Relations Director Dianne Anderson (letter attached) are interested in featuring our research outcomes in news stories to raise awareness about untrustworthy social media content. The reports would be posted in the UNM newsroom which provides information to the university and news content to print and broadcast media in state of New Mexico and beyond. The UNM newsroom site (news.unm.edu) will also promote PI Mueen as an expert in social media analytics and will respond to any media request regarding this research.

\item [O2] We will present social media data as a resource to the participants in the New Mexico Supercomputing Challenge in 2017 and onwards (letter attached). This event gathers the largest number of high school students interested in computing in the state \cite{NMSC}. We will help participants build project proposals and guide them through the solution development processes. Example projects include spotting behavioral patterns in bots, predicting trolls' activities, etc.

%We will share our methods and results online for the community, ensuring all data/code is available online in perpetuity. Additionally, t

\item [O3] The Google Play review team, which is consulting us on data gathering, is interested in testing our methods on their review data. The team intends to hire one of PI Mueen's graduate students as an intern to perform a study on the efficiency and effectiveness of our methods evaluated against their standards in their systems. While the Google Play review team is very enthusiastic about this collaboration, they were unable to provide a letter of collaboration in time due to the time needed for their legal department to conduct a review. 
%However, we expect to publish  in the required format due to legal constraints.
%Google reserves the rights of publishing the results in any format.
\end{itemize}
\begin{wrapfigure}{r}{0.3\textwidth}
\vspace{-1.5cm}
\begin{center}
\includegraphics[width=2.0in]{Figures/Poster.jpg}
\caption{Data mining projects are presented in the yearly poster session.}
\label{fig:Outreach}
\end{center}
\vspace{-0.5cm}
\end{wrapfigure}
\paragraph{Tutorials:} PI Mueen has delivered very successful tutorials in the past (see \cite{Page} for audience feedback). PI Mueen presented two three-hour tutorials at ICDM 2014 and SDM 2015 on finding repeated patterns in time series data. He tutored a half-day hands-on tutorial at the UNM Cyberinfrastructure day on data-driven scientific practices to attract researchers to data mining. To highlight the results of the proposed research, PI Mueen will create two new tutorials:
\vspace{-0.2cm}
\begin{itemize}[noitemsep,leftmargin=*]
\item [T1] A tutorial on ``Identifying Harmful Social Bots in Real Time'' proposed to KDD and/or WWW in 2019 based on the results of Task 3. 
\item [T2] A tutorial on ``Invariances in Time Series Mining: Algorithms and Applications'' proposed to ICDM 2020 based on the results of Task 1.
\end{itemize}

\begin{wrapfigure}{r}{0.3\textwidth}
\vspace{-2.9cm}
\begin{center}
\includegraphics[width=2.0in]{Figures/Outreach.jpg}
\caption{{\scriptsize Dr. Mueen mentoring a high school student and his parent on STEM careers while representing CS@UNM in a university-wide outreach event.}}
\label{fig:Outreach}
\end{center}
\vspace{-1.0cm}
\end{wrapfigure}
\paragraph{Research Projects for Undergraduate Students}
We will create small projects to maintain and update our existing data collection and storage tools. Each of these projects will be assigned among qualified undergraduate students to provide significant practical training in addition to course curricula. We will develop four projects. The first project will be on designing and implementing a {\it database in the cloud}. The second project will be on {\it monitoring new streaming APIs} based on keywords and user accounts. The third project will be on {\it building the web-portal} to disseminate our research results. Finally, the fourth project will be on using software tools for {\it sentiment classification and topic discovery} from text. PI Mueen will create specifications to set evaluation criteria and mentor students to completion. He intends to leverage NSF's REU program. Stipend funds have been included in the budget which can be covered by REU supplements in the future.




\paragraph{Involving Underrepresented Groups:} The University of New Mexico has an extremely diverse population with 45\% Hispanic, 42\% Non-minority and 10\% American Indian \cite{UNMDiv}.
%New Mexico is an EPSCoR state and UNM is one of only two universities in the nation that is both a minority-serving institution (MSI) and a Carnegie Very High Research Activity university.
The PI is anxious to exploit this demographic opportunity to involve a very large fraction of underrepresented groups in this project. The following are highlights of concrete plans for leveraging this unique opportunity to engage underrepresented groups in STEM research.

\vspace{-0.3cm}
\begin{itemize}[noitemsep,leftmargin=*]

%for this because all of the aforementioned programs subsidize or fully support the student.

\item [U1] Data mining and web mining conferences organize public contests on real problems with real data. Such contests are typically harder than class projects and suitable for advanced students of data mining. PI Mueen led a group of six graduate students for the Microsoft Malware Challenge \cite{Malware} and ranked high in the leader-board. We propose to create a {\it contest club} to continue building problem solving skills focusing on data mining and machine learning techniques. The club will be open to Masters and PhD level students and PI Mueen will supervise the club to attend at least one major contest in Kaggle in every fall semester.


\item [U2] The PI and his lab have significant industry experience in the form of full-time employment and internships. PI Mueen worked full time at Microsoft's Cloud and Information Services lab and Bing's Core Ranking team. Students of this project are good candidates for jobs in social media companies. We propose to create a monthly {\it brown-bag lunch series} for job-seeking students to discuss career planning, programming questions, programming contests and resume writing among others. 
\end{itemize}


\section{Broader Impact} 

%To people
\textbf{Societal impact:} at least 1.4 billion people on earth use social media. This staggering number becomes more comprehensible as connectivity to social media sites are now free in many of the underdeveloped countries \cite{Zuckerberg}. Such a massive body of users has diverse trustingness with essentially no help to know what can be trusted. Our techniques will help them set realistic expectations about certain users, products, hotels and sites. Our tangible outcomes will directly help them gauge trustworthiness of content and users in social media.
%To economy
\textbf{Economic impact:} social media sites support several trillion dollar industries including entertainment and tourism. Trustworthy social media is key to sustainability of these industries. For example, our work can directly impact the travel and tourism industry, which in the United States, generated nearly \$1.5 trillion in economic output in 2012. An estimated 35\% of that output was from the hospitality (e.g. hotels) and food services industry (e.g. restaurants) \cite{hospitality} that we have considered in our preliminary work. 
%Timeliness
\textbf{Timeliness:} the impact is larger than ever. While many sites (such as Pinterest) are still growing their networks, some sites (such as Twitter) are close to a decade old and have already acquired a stable network.
Such sites are more vulnerable to account merchants, spammers, trolls, cyber-attacks, etc. These sites also archive significantly long time series data containing rich patterns corresponding to events like natural disasters, political events, sports events, etc. Therefore, it is time to look at the temporal patterns of abusive activities now more than ever.
%Now is the time to look at such archive.
%To research community
\textbf{Impact to research community:} temporal pattern mining is well appreciated by scientists of various disciplines for its interpretability. Our proposed algorithmic techniques are interpretable and not limited to social media data only. The techniques can be applied to any scientific dataset with trivial modifications. In addition, our data will be shared with the research community to build better methods for abuse and inconsistency detection.

%To industry
%Social media sites can use our detection results to improve themselves. Sites can suspend bots, red-flag reviewers, warn business owners to improve their quality.




\paragraph{Prior NSF Support}
%PI Mueen has no prior NSF support.

PI Mueen: CCF-1527127, \$244,582, 10/1/15-9/30/18, \textit{CCF:SHF:Small:Collaborative Research: Domain-specific Reconfigurable Processor for Time-Series Data Mining and Monitoring} (PI: P. Brisk, Co-PI: A. Mueen). The project is recommended for funding and expected to start in October 2015.

%Intellectual Merit: This project exploits intrinsic properties of data to significantly extend the battery lifetime (and therefore usability) of domain-specific reconfigurable processors. This requires a detailed understanding of medical telemetry monitoring algorithms and representations, the strengths and limitation of reconfigurable hardware, along with issues relating to medical “culture” and human computer interaction. Broader Impact: 




\begin{figure*}[htp]
\vspace{-0.2cm}
\begin{center}
\includegraphics[width=6.6in]{Figures/Plan.pdf}
\vspace{-0.7cm} 
\caption{Timeline: Darkness of a cell refers to the level of effort. Tasks are intermingled based on relevance and dependency. Data processing will be continued throughout the duration. Summers are intensive as we have budgeted.}
\label{fig:plan}
\end{center}
\end{figure*} 

